\begin{itemize}
    \item 임의의 가산 집합 $\Omega$에 대하여, $\Omega$의 모든 부분집합의 집합 $\mathcal{P}(\Omega)$ 는 $\sigma$-대수입니다. 이제, 여기에 집합 $S$의 측도 $\mu(S)$ 를 $\abs{S}$ 라고 정의하면, $\mu$는 측도가 됩니다. 이를 \textbf{counting measure}\footnote{한국어 번역은 \textbf{셈측도} 인 것 같습니다만, 용어가 너무 어색해서 쓰지는 않을 것입니다} 라고 합니다. 
\end{itemize}